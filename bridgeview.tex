\documentclass{beamer}
\usepackage{geometry}
\usepackage[utf8]{inputenc}
\usepackage[T1]{fontenc}
%\usepackage{graphicx}
\usepackage{hyperref}
\usepackage{datetime}

\usepackage{todonotes}

\usepackage[british]{babel}

\usetheme{metropolis}

\title{Generate 3D models for large-scale objects}
\author{Jaan Toots \and Viorica Patraucean}
\institute[CSIC]{Cambridge Centre for Smart Infrastructure and
  Construction}
\date{\formatdate{24}{10}{2016}}

\begin{document}

\maketitle

\section{Introduction}

\begin{frame}{Deep learning for semantic and geometric segmentation}
  \begin{figure}[h]\centering
    \only<1>{\missingfigure{Photo of bridge}}
    \only<2>{\missingfigure{Photo of model}}
  \end{figure}
  
\end{frame}
\note{The project aims at generating BIM (Building Information Model)
  for existing bridges in an automatic way. This requires modelling
  the geometry and the materials of the bridge components. Material
  recognition from images is (relatively) solved. The challenge is on
  the geometry side.}
\note{\ldots which amounts to recognising for each bridge part what it
  represents and what shape it has, e.g.\ a column that has
  cylindrical shape. Once this info is available, a geometric fitting
  step is needed to get the parameters for each part. Due to the
  recent success of deep learning in various tasks like image
  labelling, object detection, semantic segmentation, we decided to
  use a deep learning based approach as well.}

\begin{frame}{Neural networks}
  \begin{figure}[h]\centering
    \def\layersep{2cm}

\begin{tikzpicture}[shorten >=1pt,->,draw=black!50, node distance=\layersep]
  \tikzstyle{every pin edge}=[shorten <=1pt]
  \tikzstyle{neuron}=[circle,fill=black!25,minimum size=17pt,inner sep=0pt]
  \tikzstyle{input neuron}=[neuron, fill=green!50];
  \tikzstyle{output neuron}=[neuron, fill=red!50];
  \tikzstyle{hidden neuron}=[neuron, fill=blue!50];
  \tikzstyle{annot} = [text width=4em, text centered]

  % Draw the input layer nodes
  \foreach \name / \y in {1,...,3}
  % This is the same as writing \foreach \name / \y in {1/1,2/2,3/3,4/4}
  \node[input neuron, pin={[pin edge={<-}]left:}] (I-\name) at (0,-\y) {};

  \node[annot, above of=I-1, node distance=1cm] (il) {\scriptsize input layer};

  % Draw the hidden layer nodes
  \foreach \name / \y in {1,...,4} \path[yshift=0.5cm] node[hidden
  neuron] (H1-\name) at (\layersep,-\y cm) {};

  \node[annot, above of=H1-1, node distance=1cm] (h1l) {\scriptsize hidden layer 1};
  
  \foreach \name / \y in {1,...,4} \path[yshift=0.5cm] node[hidden
  neuron] (H2-\name) at (2*\layersep,-\y cm) {};

  \node[annot, above of=H2-1, node distance=1cm] (h2l) {\scriptsize hidden layer 2};

  \foreach \name / \y in {1,...,4} \path[yshift=0.5cm] node[hidden
  neuron] (H3-\name) at (3*\layersep,-\y cm) {};

  \node[annot, above of=H3-1, node distance=1cm] (h3l) {\scriptsize hidden layer 3};

  % Draw the output layer node
  \foreach \name / \y in {1,...,2} \path[yshift=-0.5cm] node[output
  neuron, pin={[pin edge={->}]right:}] (O-\name) at
  (4*\layersep,-\y cm) {};

  \node[annot, above of=O-1, node distance=1cm] (ol) {\scriptsize output layer};

  % Connect nodes
  \foreach \source in {1,...,3}
  \foreach \dest in {1,...,4}
  \path (I-\source) edge (H1-\dest);

  \foreach \source in {1,...,4}
  \foreach \dest in {1,...,4}
  \path (H1-\source) edge (H2-\dest);

  \foreach \source in {1,...,4}
  \foreach \dest in {1,...,4}
  \path (H2-\source) edge (H3-\dest);

  \foreach \source in {1,...,4}
  \foreach \dest in {1,...,2}
  \path (H3-\source) edge (O-\dest);
\end{tikzpicture}

  \end{figure}
\end{frame}

\end{document}
